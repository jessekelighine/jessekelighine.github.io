\documentclass{article}
\usepackage{setspace}\setstretch{1.25} % {\setstretch{1.25} \par}
\usepackage[a4paper,height=25cm,hmargin={2.2cm,2.2cm}]{geometry}
\usepackage{settings}
%<--------------------------------------------------------------------------->%

\begin{document}

\centerline{%
	\large\textsf{你能通過牛津大學的數學面試嗎?}%
	\footnote{%
		作者:\texttt{jessekelighine.com}。
		靈感來自
		\href{https://www.youtube.com/channel/UCoxcjq-8xIDTYp3uz647V5A}{Numberphile}
		頻道的\href{https://www.youtube.com/watch?v=VZ25tZ9z6uI}{影片}:
		\url{https://www.youtube.com/watch?v=VZ25tZ9z6uI}。
	}
}

\begin{multicols*}{2}

在學習數學的路程中,
大家或多或少都會覺得數學只是學一些運用數字、符號的技巧,
但數學的精神實在於找尋規律與邏輯的思考,
% 數學的精神在於找尋規律與邏輯的思考,
牛津大學的數學面試就充分地體現這樣的想法:
\href{https://www.seh.ox.ac.uk/people/tom-crawford}{Tom Crawford}
教授設計的其中一個面試問題中就沒有任何計算,
並且問題敘述簡單到連小學、國中生都可以懂。
這樣的問題不是要看出一個學生會多少複雜的數學,
而是要測驗他們探索現象的方式以及歸納與推演的能力。
我們就來看看你可不可以通過牛津大學的數學面試吧!

\textsf{問題}:
可以用 $n$ 個長寬比 $2\ratio1$ 的長方形鋪滿一個正方形嗎?
為了方便,我們令正方形的邊長為 $1$。
顯然地 $n$ 不能等於 $1$,因為正方形的長寬比是 $1\ratio1$;
但是 $n=2$ 就是可行的,方式如下:

\putfig{case-two}

\noindent
嚴謹一點說:給定任意的正整數 $n$,
是否可以用 $n$ 個長寬比為 $2\ratio1$ 的長方形來鑲嵌一個正方形?
注意到問題中不要求使用的長方形大小要相同,只要長寬比是 $2\ratio1$ 就 ok。
動手試試看吧!

\separationline

首先,
觀察到我們可以把一個鑲嵌中的一個長方形分成 $2$ 個小正方形,
再各用兩個 $2\ratio1$ 的長方形鑲嵌那 2 個小正方形。
把範例 $n=2$ 鑲嵌中的其中一個長方形以這樣的方式分割會得到:

\putfig{case-demo}

\noindent
這是一個重要的觀察,因為藉由這樣的分割過程,
如果我們知道 $n$ 個長方形的鑲嵌是可行的,就可以推論 $n+3$ 也是可行的;
同理 $n+6$、 $n+9$、⋯ 都是可行的,只要一直分割下去就可以了。
如此一來既然我們已經知道 $n=2$ 是可行的,
任何符合 $2+3k$ 形式的正整數都是可行的。
如果列出幾個正整數並且用綠色、紅色分別標示可行、不可行的數字,
會有這樣一個表:

\putfig{family-1}

\noindent
有趣的我們乎就找到了一整個可行的正整數「家族」,
其中的成員都是由 $n=2$ 這個「鑲嵌原型」所生成的。
我們可以把任何一個家族都表示成「$n+3k$」,
其中 $n$ 就是鑲嵌原型中所使用的長方形數。

雖然我們找到了一個有無限多個成員的家族,
但還是有很多數字無家可歸。
那麼是不是存在其他家族以及原型呢?
在建立 $n=2$ 的鑲嵌時,我們把正方形水平的切成一半;
但如果把正方形的其中一邊切成更多等分,可能就可以找到其他的家族。
既然已經分成過 2 等份了,
我們可以試著把其中一邊分成 3 等份,這樣就會有以下的鑲嵌:

\putfig{case-six}

\noindent
這樣的鑲嵌總共用了 $6$ 個長方形,是一個不在 $2+3k$ 家族中的數字,
可以生成另一個新的家族: $6+3k$。
有了這樣的新作法後,我們可以嘗試更多的等分;
為了避免找之前已經找到的數字,
我們跳過 4 等分直接考慮把分成 5 等分的情形:

\putfig{case-ten}

\noindent
發現這是 $n=10$ 的鑲嵌,可以生成家族 $10+3k$。
這樣的方法當然可以一直做下去,
但其實多找到這兩個家族就已經足夠了。
我們更新一下剛剛寫下來的表:

\putfig{family-2}

\noindent
注意到除了我們找到的三個家族之外,
出現了連續三個都可以達成的數:8、9、10。
由於我們知道「如果 $n$ 是可行的,$n+3$ 就是可行的」,
有了連續三個可行的數就可以推論所有大於這三個數的整數都是可行的。
如此一來我們就一口氣解決了所有大於等於 8 的正整數,
並且知道這些數字都是可行的!
那麼剩下的工作就是考慮比 8 小且還沒被檢驗的數字了:3、4 和 7。

先從最小的數字下手。
多試幾次之後會發現 $n=3$ 似乎不太可能,
那我們就試著證明 $n=3$ 是不可行的。
其中一個可能的思路是這樣的:
一個正方形有 4 個角落,如果只有 3 個長方形可以用,
就一定要有一個長方形要佔據 2 個角落。
既然要佔據 2 個角落,
那麼那個長方形的長就必須是 1、寬是 $1/2$:

\putfig{case-three}

\noindent
但是顯然地剩下的部分無論水平、垂直的切,
都不可能把剩下的部分切成兩個 $1\ratio2$ 的長方形,
這麼一來我們就證明了 $n=3$ 是不可能達成的。

有了這樣的證明思路之後,
會發現 $n=4$ 的形情也可以用類似的方式證明。
先考慮有沒有可能讓一個長方形佔據 2 個角落:
這樣就代表剩下的部分必須由 3 個長方形填滿,
循著剛剛 $n=3$ 的證明思路就會發現這是不可能的。
如果不能有長方形佔據 2 個角落的話,
剩下的可能就是 4 個長方形各佔 1 個角落了:
如果我們先考慮 2 個相鄰角落的話,
顯然不可能這 2 個長方形都是「站著」的
(因為這樣會變成 $n=2$ 的情形),
至少要有一個是「躺著」的:

\putfig{case-four}

\noindent
但是顯然地,如果拉動中間的虛線並同時保持 2 個長方形的比例,
剩下的部分不可能被兩個 $1\ratio2$ 的長方形填充。
(細節就請讀者自己完成吧!)
如此一來我們也就成功地證明了 $n=4$ 也是不可行的。

那 $n=7$ 的情況呢?
這個情況用上述的思路證明有點困難,
況且我們也不是很確定 $n=7$ 是不是真的不可行。
難道要再找另外一個原型嗎?
如果我們再去看看列表的話,
會發現 $10+3k$ 這個家族是從 $n=10$ 的開始的,
如果我們可以讓這個家族從 $n=7$ 開始呢?
換句話說,是不是可以把 $n=10$ 的鑲嵌中 4 個長方形何在一起變成一個長方形呢?

\putfig{family-3}

有了這樣的想法之後答案就呼之欲出了:
我們可以把 $n=10$ 的鑲嵌中左上角的 4 個長方形合併,
得到 $n=7$ 的鑲嵌了:

\putfig{case-seven}

\noindent
$n=7$ 的例子之所以這麼難找大概就是因為這個鑲嵌是我們找的所有鑲嵌原型中,
唯一需要 3 種不同大小的長方形的鑲嵌。

如此一來我們的就可以大方的把答案寫出來了:

\putfig{family-4}

\noindent
除了 1 明顯不行以及 3 跟 4 因為太小所以不可行之外,
所有其他的正數都是可行的!
你看到題目時有猜到這個令人意外的結果嗎?

\separationline

在 \href{https://www.youtube.com/watch?v=VZ25tZ9z6uI}{Numberphile 的影片}中
Tom Crawford 教授也使用了類似的解決思路,
這邊只是提供一種最後可以比較容易解決 $n=7$ 的思路。
但無論哪種推論方式,
以下幾個重要概念都少不了:
\begin{itemize}
	\item
		發現遞迴關係「$n$ 可行 $\implies$ $n+x$ 可行」。(本文中提供 $x=3$ 的思路)
	\item
		認知到這代表一整個數列都可行。
	\item
		從以上的啟示尋找其他的數列、家族。
	\item
		透過模運算、歸納⋯等方式發現到連續 $x$ 個可行整數代表所有更大的整數都可行。
	\item
		歸謬證明 3、4 個情形不可行。
	\item
		最後加上一點點運氣發現 $n=7$ 可行。
\end{itemize}
這個題目精巧之處在於解題的過程中可以只用國中、小學生都可以懂的數學知識,
不必透過複雜的計算或技巧就可以檢驗學生對於這些數學原理的理解、靈活運動的能力。
同時這個問題也可以有各式各樣的延伸:
「如果把長方形的長寬比換成 $3\ratio1$ 呢?」、
「我們可以把『家族』定義得更嚴謹嗎?」、
「可以找到所有的家族嗎?」⋯等,
都是非常有趣、值得繼續思考與探討的問題。
\hfill$\blacksquare$

\vfill

\hfill{2021 年 8 月}

\end{multicols*}

\end{document}
