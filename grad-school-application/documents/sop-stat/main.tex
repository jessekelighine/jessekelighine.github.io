\documentclass[a4paper]{article}
\usepackage{fontspec}\defaultfontfeatures{Ligatures=TeX}
\usepackage{setspace}\setstretch{1.1} % \begin{spacing}{1.3}
\usepackage[a4paper,vmargin={3.5cm,3.5cm},hmargin={3.5cm,3.5cm}]{geometry}
\usepackage[skip=2pt plus1pt,indent=2em]{parskip}
%-----------------------------------------------------------------------------%
%%% Page # of # %%%
\usepackage{fancyhdr,lastpage}
\pagestyle{fancy}\renewcommand{\headrulewidth}{0pt}\fancyhf{}
\fancyfoot[C]{\small Page \thepage\ of \pageref{LastPage}}
\fancypagestyle{plain}{\renewcommand{\headrulewidth}{0pt}\fancyhf{}
\fancyfoot[C]{\small Page \thepage\ of \pageref{LastPage}}}
%-----------------------------------------------------------------------------%
%%% Font %%%
\setmainfont{EB Garamond}
\setmonofont{IBM Plex Mono}
%-----------------------------------------------------------------------------%
%%% xeCJK %%%
\usepackage{xeCJK}
\setCJKmainfont{GenWanMin TJ L}
%-----------------------------------------------------------------------------%

\begin{document}

\noindent
\textbf{Personal Statement}
\hfill
\textsc{Applicant: Jesse Chieh Chen}\\
\null\hfill{\footnotesize\texttt{r11323013@ntu.edu.tw}}\\
ETH Z\"urich MSc Statistics

\unskip
\vspace{0.3em}
\hrule
\vspace{1em}

\noindent
Dear Members of the Admissions Committee,\\

\noindent
I am writing to express my interest in pursuing a MSc in Statistics at ETH Z\"urich.
\unskip\
In 2018, I started my journey in economics at National Taiwan University (NTU).
% My keen interest in statistics and econometrics led me to start pursuing a minor in mathematics during the following year.
My passion for quantitative analysis prompted me to pursue a minor in mathematics in the following year.
I am proud to have received four Dean's List awards (top 5\%) during my undergraduate studies.
% Due to Covid19 and conscription considerations,
% I pursued an MSc in Economics after my bachelor's degree to keep my studies going in preparation for future studies abroad.
Amidst the challenges posed by Covid-19 and the need to fulfill conscription requirements,
I started an MSc in Economics following my bachelor's degree,
with the intent of maintaining academic continuity in preparation for future overseas studies.
I am well-positioned to graduate from my master's program with the highest GPA in my graduating class.
During my studies, my interest shifted from economics to statistics as I gained theoretical understanding
and hands-on experience of econometrics and statistics.
I aim to enhance my statistical abilities and gain a deeper understanding,
as well as practical skills, throughout my MSc programme in statistics.

The first research experience I had was working as an RA under Professor Ming-Jen Lin.
We worked with the large Taiwanese health insurance data to investigate intergenerational health transmission.
I was involved in reconstructing family trees using household registry information,
devising and performing hypothesis testing, and running structural estimation using R and Stata.
This experience provided me with a solid foundation in programming and statistical inference in practice.

% Working with this large data set prompted me to think about how health issues spread through networks such as family structures,
% socio-economic strata, and geographic locations.
% As I gained experience in the workflow of empirical studies,
% my interest shifted towards the theoretical and econometric aspects of estimating social-interaction effects.

Subsequently, I started RA work with Professor Hon-Ho Kwok on estimating spatial auto-regressive (SAR) models.
My responsibilities primarily involved implementing the estimation process in Matlab.
This collaboration culminated in my bachelor's thesis,
where I focused on estimating SAR models using non-linear least squares (NLS).
Aided by my solid mathematical foundation in analysis, linear algebra, and probability,
I demonstrated NLS's consistency and asymptotic normality.
And with my experiences in computation, I validated these results through extensive numerical simulations in R
and proposed a feasible two-step estimation procedure to achieve optimal asymptotic efficiency.

Currently, I am tackling the challenge of unknown networks in SAR models, a high-dimensional problem.
After learning about Bayesian statistics and related computational methods from Professor Chih-Sheng Hsieh,
I realized that in comparison to the frequentist approach,
a Bayesian approach is more suitable for estimating unknown networks.
My current master's thesis revolves around establishing a Bayesian method for estimating unknown networks.
I've developed preliminary methods for sampling from random matrices and
conducting statistical inference using Gibbs sampling and Markov Chain Monte Carlo.
Having completed my conscription duty in the latter part of 2023,
I plan to finish my masters thesis and graduate in June 2024.

In the future, I aspire to become a statistician.
As my interest shifted from economics to statistics,
I wish to pursue more in to the field of statistics both in theory and practice.
LSE provides an excellent statistics programme that covers a vast array of topics
that I had not had the opportunity to delve into in depth.
I believe that my solid foundation in mathematics and a strong focus on econometrics/statistics and related computational methods
are well-suited for MSc in Statistics at LSE.
\unskip

\vspace{1em}
\noindent
Thank you for considering my application.\\

\vspace{2em}
\hfill
\begin{minipage}{.35\textwidth}
	Sincerely,\\
	Jesse Chieh Chen\hspace{0.5em}陳\,捷
\end{minipage}

\end{document}
