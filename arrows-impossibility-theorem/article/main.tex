%!TEX program = xelatex
\documentclass[a4paper]{article}
\usepackage{fontspec}\defaultfontfeatures{Ligatures=TeX}
% \usepackage{setspace}\setstretch{1.3} % \begin{spacing}{1.3}
\usepackage[a4paper,vmargin={0.8cm,1cm},hmargin={1cm,1cm}]{geometry}
%-----------------------------------------------------------------------------%
\usepackage{settings}
%-----------------------------------------------------------------------------%
%%% Title %%%
\title{Arrow's Impossibility Theorem: A One-Page Summary%
\thanks{This note draws heavily from \textcite{maskin-sen-2014} \citetitle{maskin-sen-2014} and
\textcite{rubinstein-2006} \citetitle{rubinstein-2006}. This note is essetionally a quick summary
of some of the results mentioned in both for my own reference.}%
}
\author{\href{https://jessekelighine.com}{jessekelighine.com}\\Jesse C. Chen 陳捷}
\date{\today}
%-----------------------------------------------------------------------------%

\begin{document}

\maketitle

\begin{multicols}{2}

\section{Statement of the Theorem}

\begin{definition}[Preference Relation]
	Let $\mathcal{X}$ denote a set of alternatives.
	A binary relation $\succsim$ over $\mathcal{X}$ is said to be a \emph{preference relation}
	if it satisfies the following properties:
	\begin{enumerate}
		\item Completeness:
			$\forall \{x,y\}\subseteq\mathcal{X}$, $x\not\succsim y$ implies $y\succsim x$.
		\item Transitivity:
			$\forall \{x,y,z\}\subseteq\mathcal{X}$, $x\succsim y$ and $y\succsim z$ implies $x\succsim z$.
	\end{enumerate}
\end{definition}

\begin{definition}[\gls*{scm}]
	An \gls*{scm} consists of the following:
	\begin{enumerate}
		\item \textsc{Alternatives}:
			A set of alternatives $\mathcal{X}$.
		\item \textsc{Individuals}:
			A set of finite individuals $\mathcal{N}$ with $|\mathcal{N}|=n$.
		\item \textsc{Preference Profile}:
			An $n$-tuple of individual preferences $P=(\succsim_1,...,\succsim_n)$ over alternatives $\mathcal{X}$.
			Denote the space of all possible preference profiles as $\mathcal{P}$.
		\item \textsc{\glsentrylong{swf}}:
			A \gls*{swf}, denoted by $\varphi$,
			is a function that assigns a preference relation $\succsim_P$ to every possible preference profile $P$.
			We write the output of $\varphi(P)$ as $\succsim_{P}$.
	\end{enumerate}
\end{definition}

\begin{remark}
	Something to note with \gls*{swf}:
	\begin{enumerate}
		\item
			\gls*{swf} is a way to aggregate preference to form an ``over-all preference'' of the individuals.
			Note that this means for any profile $P$,
			$\varphi$ cannot simply spit-out a single element in $\mathcal{X}$ and declare it to be `the' choice.
			The output has to be a preference relation,
			which is able to compare every element in the alternatives $\mathcal{X}$.
		\item
			We require \gls*{swf} to be a function,
			i.e., it has to produce a social preference for any possible preference profile.
			This properties is sometimes referred in the literature as \emph{unrestricted domain}.
	\end{enumerate}
	Naturally, we wish an \gls*{swf} to satisfy certain properties we deem ``fair'' or ``obvious.''
\end{remark}

\begin{axiom}[\gls*{par}]
	An \gls*{swf} is said to satisfy \emph{\gls*{par}}
	if for any profile $P=(\succsim_i)_{i\in\mathcal{N}}$ such that $x\succsim_i y$ $\forall i\in\mathcal{N}$,
	we have $x\succsim_{P} y$.
\end{axiom}

\begin{remark}
	That is to say, if everyone prefers $x$ to $y$,
	then the aggregated social preference should also reflect that.
\end{remark}

\begin{axiom}[\gls*{iia}]
	An \gls*{swf} is said to satisfy \emph{\gls*{iia}}
	if for any two profiles $P=(\succsim_i)_{i\in\mathcal{N}}$ and $P'=(\succsim_i')_{i\in\mathcal{N}}$
	such that $x\succsim_i y$ iff $x\succsim_i' y$,
	we have $x\succsim_P y$ iff $x\succsim_{P'}y$.
\end{axiom}

\begin{remark}
	That is, if two preference profiles ranks alternatives $x$ and $y$ the same way,
	then the aggregate preferences should rank $x$ and $y$ the same way,
	regardless of how other alternatives are ranked in the two profiles.
\end{remark}

\begin{theorem}[Arrow's Impossibility Theorem]
	Under an \gls*{scm} with $|\mathcal{X}|\geq3$,
	any \gls*{swf} $\varphi$ that satisfies \gls*{par} and \gls*{iia} is dictatorial,
	that is, $\exists i^*\in\mathcal{N}$ such that $\varphi(P)=\mathord{\succsim_{i^*}}$ $\forall P\in\mathcal{P}$.
\end{theorem}

\section{Proof of the Theorem}

\begin{definition}[Decisiveness]
	Under an \gls*{scm},
	a group of individuals $\mathcal{G}\subseteq\mathcal{N}$ is said to be
	\emph{decisive} over a pair of alternatives $\{x,y\}\subseteq\mathcal{X}$
	if $x\succsim_{P} y$ whenever preference profile $P$ satisfies $x\succsim_i y$ $\forall i\in\mathcal{G}$.
\end{definition}

\begin{lemma}[Globally Decisive Group]\label{lem:globally-decisive-group}
	Suppose an \gls*{swf} satisfies \gls*{par} and \gls*{iia}.
	If a group $\mathcal{G}$ is decisive over a pair $\{x,y\}\subseteq\mathcal{X}$,
	then the group is decisive over every pair in $\mathcal{X}$.
\end{lemma}

\begin{proof}
	Let $\{a,b\}\subseteq\mathcal{X}$ be different from $\{x,y\}$.
	Suppose that in a certain profile $P$, we have $a\succsim_i x$ and $y\succsim_i b$ $\forall i\in\mathcal{N}$.
	By \gls*{par}, we must have $a \succsim_P x$ and $y\succsim_P b$.
	Further suppose that in profile $P$, we have $x\succsim_i y$ $\forall i\in\mathcal{G}$.
	Then, since $\mathcal{G}$ is decisive over $\{x,y\}$, we have $x\succsim_P y$.
	Since $\succsim_P$ must be transitive, we have $a\succsim_P b$.
	Notice that for the individuals outside of $\mathcal{G}$,
	the preference relation between $a$ and $b$ is unspecified under $P$.
	Consider another preference profile $P'$ where the preference relation between $\{a,b\}$
	is the same as $P$ for all individuals,
	but the preferences over other alternatives, including $\{x,y\}$, are arbitrary.
	By \gls*{iia}, we must have $a\succsim_{P'} b$ since $a\succsim_P b$.
	Hence, the group $\mathcal{G}$ is also decisive over the pair $\{a,b\}$.
	Similarly, we can consider pairs $\{x,b\}$ or $\{a,y\}$
	and conclude that $\mathcal{G}$ is decisive over those pairs.
\end{proof}

\begin{remark}
	By \autoref{lem:globally-decisive-group},
	decisiveness over any pair entails decisiveness over every pair.
	Hence, we will simply refer to groups as \emph{decisive} without specifying the pair of alternatives.
\end{remark}

\begin{lemma}[Contraction of Decisive Group]\label{lem:contraction-of-decisive-group}
	Suppose an \gls*{swf} satisfies \gls*{par} and \gls*{iia}.
	If $\mathcal{G}$ is decisive (with more than one individual),
	then a proper subset of $\mathcal{G}$ is also decisive.
\end{lemma}

\begin{proof}
	Partition the group $\mathcal{G}$ into two non-empty sub-groups: $\mathcal{G}_1$ and $\mathcal{G}_2$.
	Suppose that we have a profile $P$ such that
	$x\succsim_i y$ and $x\succsim_i z$ $\forall i \in\mathcal{G}_1$, also
	$x\succsim_i y$ and $z\succsim_i y$ $\forall i \in\mathcal{G}_2$.
	Since $\mathcal{G}$ is decisive, we have $x\succsim_P y$.
	Consider two cases:
	\begin{itemize}
		\item
			Suppose $z\succsim_P x$, then by transitivity we have $z\succsim_P y$.
			Notice that no assumption about the preference relation over $\{y,z\}$
			is made apart from the individuals in $\mathcal{G}_2$.
			Hence, by \gls*{iia}, $\mathcal{G}_2$ is decisive since $\mathcal{G}_2$ is decisive over $\{y,z\}$.
		\item
			Suppose $z\not\succsim_P x$, then by completeness we have $x\succsim_P z$.
			Similarly, no assumption about the preference relation over $\{x,z\}$
			is made apart from the individuals in $\mathcal{G}_1$.
			Hence, by \gls*{iia}, $\mathcal{G}_1$ is decisive.
	\end{itemize}
	Therefore, between $\mathcal{G}_1$ and $\mathcal{G}_2$, one of which must be decisive.
\end{proof}

\begin{proof}[Proof of Arrow's Impossibility Theorem]
	By \gls*{par}, all individuals as a group $\mathcal{N}$ is decisive.
	By \autoref{lem:contraction-of-decisive-group},
	we known that a proper subset of $\mathcal{N}$ is, too, decisive.
	Since $\mathcal{N}$ is finite,
	this process of `contracting' the decisive group terminates when the decisive group contains only one individual,
	the dictator.
\end{proof}

\end{multicols}

\end{document}
