\documentclass[a4paper]{article}
\usepackage{fontspec}\defaultfontfeatures{Ligatures=TeX}
\usepackage{setspace}\setstretch{1.1} % \begin{spacing}{1.3}
% \usepackage[a4paper,vmargin={4cm,4cm},hmargin={4cm,4cm}]{geometry}
%-----------------------------------------------------------------------------%
\usepackage{settings}
%-----------------------------------------------------------------------------%
%%% Title %%%
\title{Fisher Information}
\author{\href{https://jessekelighine.com}{jessekelighine.com}\\Jesse C.\ Chen\ 陳\,捷}
\date{\today}
%-----------------------------------------------------------------------------%

\begin{document}

\maketitle

\noindent
\textbf{Fisher Information} is defined as follows:
\begin{align*}
	\mathcal{I}_{\theta}
	\coloneqq
	\E
	\left[
	\left(
	\frac{\partial}{\partial\theta}\log f(x\given\theta)
	\right)
	\left(
	\frac{\partial}{\partial\theta}\log f(x\given\theta)
	\right)\T
	\given[\middle]
	\theta
	\right]
\end{align*}
where $f(\cdot\given\theta)$ is the likelihood function of the random variable $x$ with parameter $\theta$.
The expectation is taken with respect to the random variable $x$.

In this note,
we talk about what Fisher Information is,
its relation to \emph{maximum likelihood},
its conflicting intuitions,
and \emph{Cramér-Rao bound}, an inequality closely related to Fisher Information.
\footnote{
	I this note, I shall assume the knowledge of maximum likelihood estimation and linear algebra.
}

\section{What are we trying to do?}

The goal is simple:
\begin{quote}
	\textit{We want to have a way of measuring ``how informative a parameter is to the likelihood function.''}
\end{quote}
If a certain parameter is very important in shaping the likelihood function,
we expect our estimate of it to be more ``precise.''
In contrast, if a parameter does not change the shape of the likelihood much,
then we would expect the estimate of such parameter ``poor,''
since likelihood functions under different true parameters look the same to us.

We will later see that \emph{Fisher Information} achieves precisely this purpose
in the context of maximum likelihood estimation.

\section{Variance of Score Function}

Let's first define some notation.
Let $\ell(x\given\theta)\coloneqq\log f(x\given\theta)$ be the log-likelihood function.
We use Newton's notation to denote differentiation with respect to the parameter,
that is,
\begin{align*}
	\dot\ell(x\given\theta)
	\coloneqq \frac{\partial}{\partial\theta} \ell(x\given\theta)
	= \frac{\partial}{\partial\theta} \log f(x\given\theta).
\end{align*}
Notice that $\dot\ell(x\given\theta)$ has mean zero:
\begin{align*}
	\E\dot\ell(x\given\theta)
	&= \int_{\mathcal{X}} \left[ \frac{\partial}{\partial\theta} \log f(x\given\theta) \right] f(x\given\theta) dx \\
	&= \int_{\mathcal{X}} \dot f(x\given\theta) dx 
	= \frac{\partial}{\partial\theta} \int_{\mathcal{X}} f(x\given\theta) dx = 0.
\end{align*}
Thus, we have $\var\dot\ell(x\given\theta) = \E\dot\ell(x\given\theta)\dot\ell(x\given\theta)\T$,
\footnote{
	We shall assume that Leibniz rule of differentiation is always satisfied for interchanging integration and differentiation.
}
which is the definition of \emph{Fisher Information}.
This observation gives us the first intuition of Fisher Information:

\begin{intuition}\label{intuition-1}
	Large Fisher Information means large variance in the score function $\dot\ell(x\given\theta)$.
	\footnote{
		A large covariance matrix means the variance is large along every direction.
		See \href{https://jessekelighine.com/\#statistics}{here} for an intuitive explanation.
	}
	And a large variation in the score function means that our maximum likelihood estimate,
	which solves the problem
	\begin{align*}
		\arg\max_{\theta\in\Theta}\ell(x\given\theta)
	\end{align*}
	via first order condition
	\begin{align*} 
		\dot\ell(x\given\theta)=0,
	\end{align*}
	also has a large variance.
	This means that a large Fisher Information is \textbf{bad}.
\end{intuition}

\section{Information Equality}

Before jumping to conclusion,
let's consider $\E\ddot\ell(x\given\theta_0)$,
the expected Hessian matrix of the log-likelihood function.
It turns out $\E\ddot\ell(x\given\theta_0)$ and $\var\dot\ell(x\given\theta_0)$ only differ by a negative sign:
\begin{align*}
	\E\ddot\ell(x\given\theta)
	&= \int_{\mathcal{X}} \left[ \frac{\partial^2}{\partial\theta\partial\theta\T} \log f(x\given\theta) \right] f(x\given\theta) dx \\
	&= \int_{\mathcal{X}} \left[ \frac{\ddot f(x\given\theta)}{f(x\given\theta)} - \frac{\dot f(x\given\theta)\dot f(x\given\theta)\T}{f(x\given\theta)^2} \right] f(x\given\theta) dx \\
	&= \underbrace{\frac{\partial^2}{\partial\theta^2} \int_{\mathcal{X}} f(x\given\theta) dx}_{=0} - \int_{\mathcal{X}} \dot\ell(x\given\theta)\dot\ell(x\given\theta)\T f(x\given\theta) dx
	= -\var\dot\ell(x\given\theta)
\end{align*}
This result is sometimes referred to as \emph{information equality}.
The equality states that the asymptotic variance of $\hat\theta$ is simply the negative of the Hessian of the log-likelihood function.

However, this means that we can also defined \textbf{Fisher Information} as $-\E\ddot\ell(x\given\theta)$,
which yields another intuition:

\begin{intuition}\label{intuition-2}
	A large Fisher information is actually \textbf{good},
	since it is the Hessian matrix of the log-likelihood function.
	A larger Hessian implies a larger curvature,
	meaning that the log-likelihood function is more ``defined'' or ``pointy'' for our parameter,
	which naturally leads to a better estimation.
\end{intuition}

This is in direct conflict with \autoref{intuition-1},
how do we resolve this?

\section{Maximum Likelihood}

Let's be precise.
Suppose we have the data set $\{x_{i}\}_{i=1}^{n}\iidto f(\cdot\given\theta_0)$.
Then we denote the joint log-likelihood as
\begin{align*}
	\ell(\{x_i\}\given\theta) \coloneqq \sum_{i=1}^{n} \ell(x_i\given\theta).
\end{align*}
We want to derive the asymptotic distribution of the maximum likelihood estimator $\hat\theta$ as $n\to\infty$.
By mean value theorem, we have
\begin{align*}
	\underbrace{\dot\ell(\{x_i\}\given\hat\theta)}_{=0} - \dot\ell(\{x_i\}\given\theta_{0}) = \ddot\ell(\{x_i\}\given\bar\theta) (\hat\theta - \theta_0)
\end{align*}
where $\bar\theta$ is a value between $\theta_0$ and $\hat\theta$.
Supposing we already have the consistency result $\hat\theta\pto\theta_0$,
we have
\begin{align*}
	\sqrt{n} ( \hat\theta - \theta_0 )
	&= - \left[ \frac{ \ddot\ell(\{x_i\}\given\bar\theta) }{ n } \right]\inv \left( \frac{ \dot\ell(\{x_i\}\given\theta_{0}) }{ \sqrt{n} } \right) \\
	&\dto - \E\left[\ddot\ell(x\given\theta_0)\right]\inv \mathcal{N}(0,\var\dot\ell(x\given\theta_0)).
\end{align*}
Now we know the asymptotic distribution of $\sqrt{n}(\hat\theta-\theta_0)$ is a normal distribution centered at $0$ with variance
$\E[\ddot\ell(x\given\theta_0)]\inv\var\dot\ell(x\given\theta_0)\E[\ddot\ell(x\given\theta_0)]\inv$.

Notice this this result confirms with both of our intuitions:
we have $\var\dot\ell(x\given\theta)$ in the nominator (in concord with \autoref{intuition-1}),
and $\E\ddot\ell(x\given\theta_0)$ in the denominator (in concord with \autoref{intuition-2}).
Hence, it is a trade-off between these two effects we found in the two intuitions.
And by \emph{information equality}, we obtain the famous maximum likelihood result:
\begin{align*}
	\sqrt{n} ( \hat\theta - \theta_0 ) \dto \mathcal{N}(0,\mathcal{I}_{\theta}\inv).
\end{align*}
Therefore, a \textbf{``larger''} Fisher Information is what we really want.
This does not mean that \autoref{intuition-2} is correct and \autoref{intuition-1} is wrong,
it is a result of both intuitions combined.

Note that the maximum likelihood result achieves what we set out to do in the first place.
We now know that the estimation quality of $\hat\theta$ depends on the size of $\mathcal{I}_{\theta}$:
the larger $\mathcal{I}_{\theta}$ is, the more precise the estimation,
i.e., under the same number of samples $n$, a larger $\mathcal{I}_{\theta}$ yields a more precise result.
\footnote{
	In some contexts (mostly in Bayesian statistics),
	the inverse of a covariance matrix is called the \emph{precision matrix}.
	Thus, we can view Fisher Information as the precision matrix of the asymptotic distribution.
	I find this name particularly fitting in this context.
}

\section{Cramér-Rao Bound}

Now we know that Fisher Information tells us the precision of our estimation,
but can we do better?
Can we be more precise?
The answer is \textbf{NO}, given that we want the estimation to be unbiased.
\footnote{
	Maximum likelihood estimation is not always unbiased,
	but the order of the bias is small $O(1/n)$.
}
This result is called \emph{Cramér-Rao bound},
stating that any other unbiased estimation of $\theta_0$ must have variance larger than $\mathcal{I}_{\theta}\inv$.

Suppose $\tilde\theta=t(\{x_i\})$ is an unbiased estimator for $\theta_0$,
it's straightforward via some algebra to see that
we have the covariance between $\dot\ell(\{x_i\}\given\theta)$ and $t(\{x_i\})-\theta$ as an identity matrix of dimension $k$:
\begin{align*}
\E [t(\{x_i\})-\theta] \dot\ell(\{x_i\}\given\theta)\T
= I_k,
\end{align*}
where $k$ is the length of $\theta_0$.

A slight variation of Cauchy-Schwarz inequality states that
for random vectors $v$ and $u$ both with mean $0$, we have that
$\E(vv\T) - \E(vu\T)\E(uu\T)\inv\E(uv\T)$
is positive definite. 
\footnote{
	Note $\E(v+Au)(v+Au)\T$ is positive definite.
	Let $A=-\E{vu\T}(\E{uu\T})\inv$ and we obtain the desired result.
}
If we plugin $v=t(\{x_i\})-\theta$ and $u=\dot\ell(\{x_i\}\given\theta)$,
we have
\begin{align*}
	\var t(\{x_i\}) - (n\mathcal{I}_{\theta})\inv
\end{align*}
is positive definite.
Therefore, $\var{t(\{x_i\})}$ is larger than $(n\mathcal{I}_{\theta})\inv$.
\footnote{
	If you are not sure what does positive definiteness have to do with the concept of ``larger,''
	Click \href{https://jessekelighine.com\#statistics}{here} to see an intuitive explanation.
}

What does it mean to be compared to $(n\mathcal{I}_\theta)\inv$?
When $n$ is sufficiently large, we have the approximation
\begin{align*}
	\hat\theta - \theta_0 \overset{A}{\sim} \mathcal{N}(0,(n\mathcal{I}_\theta)\inv)
\end{align*}
Hence, we can see that the variance of our maximum likelihood estimator is approximately $(n\mathcal{I}_{\theta})\inv$,
which beats every other unbiased estimator.

\dinkus

\noindent
In a nutshell:

\begin{tcolorbox}[mycolorbox={myblue}]
	\begin{enumerate}
		\item
			\textbf{Maximum likelihood} is the best method
			in the sense that it produces the most precise estimator among all the unbiased estimators.
		\item
			\textbf{Fisher Information} is a reasonable definition of ``information,''
			as it describes the best ``precision'' under all unbiased estimators of $\theta$.
	\end{enumerate}
\end{tcolorbox}

\end{document}
