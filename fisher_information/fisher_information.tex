\documentclass[a4paper,11pt]{article}
\usepackage{fontspec}\defaultfontfeatures{Ligatures=TeX}
% \usepackage{setspace}\setstretch{1.25} % {\setstretch{1.25} \par}
\usepackage[a4paper,vmargin={4cm,4cm},hmargin={4cm,4cm}]{geometry}
%<--------------------------------------------------------------------------->%
%%% Date Time %%%
\usepackage{datetime}
\newdateformat{meinesdatum}
{\THEYEAR-\twodigit{\THEMONTH}-\twodigit{\THEDAY}}
\setdefaultdate{\meinesdatum}
%-----------------------------------------------------------------------------%
%%% Minimal Title %%%
\makeatletter\def\@maketitle{%
\noindent{\sffamily\bfseries\@title}%
\hfill\texttt{\@date}\\{\ }\null\hfill\texttt{\@author}
\par\vskip1.5em}\makeatother
%-----------------------------------------------------------------------------%
%%% Section Font %%%
\usepackage{titlesec}
\titleformat*{\section}{\normalsize\bfseries\sffamily\raggedright}
%-----------------------------------------------------------------------------%
%%% Enumerate %%%
\usepackage{enumitem}
\setlist{listparindent=\parindent,parsep=0pt,nosep} % paragraph indent 
%-----------------------------------------------------------------------------%
%%% Math %%%
\usepackage{amsmath,mathtools,amssymb,mathrsfs}
\usepackage{amsthm}
\newcommand{\dd}{\,d}
%-----------------------------------------------------------------------------%
%%% Probability %%%
\let\Pr\relax
\DeclareMathOperator{\Pr}{\text{\textrm{\upshape\textbf{P}}}}
\DeclareMathOperator{\E}{\text{\textrm{\upshape\textbf{E}}}}
\DeclareMathOperator{\var}{Var}
\DeclareMathOperator{\cov}{Cov}
\newcommand{\iidto}{\overset{\text{\scriptsize iid}}{\sim}}
\newcommand{\asto}{\overset{as}{\longrightarrow}}
\newcommand{\pto}{\overset{p}{\longrightarrow}}
\newcommand{\dto}{\overset{d}{\longrightarrow}}
\newcommand{\pr}[1]{\Pr\left\{#1\right\}}
\newcommand{\given}[1][]{\,#1|\,}
%-----------------------------------------------------------------------------%
\newcommand{\normal}[1]{\mathcal{N}\left(#1\right)}
\newcommand{\inv}{^{-1}}
%-----------------------------------------------------------------------------%
%%% Hyper Reference %%%
\PassOptionsToPackage{hyphens}{url}
\usepackage[hidelinks,colorlinks=true]{hyperref}
% \newcommand{\supref}[1]{\ref{#1} (p.\pageref{#1})}
%-----------------------------------------------------------------------------%
\title{Fisher Information}
\author{\texttt{jessekelighine.com}}
\date{\today}
%<--------------------------------------------------------------------------->%

\begin{document}

\maketitle

\noindent
The \emph{Fisher Information} is given by the following:
\begin{align}\label{eq:fisher}
	\mathcal{I}_\theta
	:= \E\left[\left(\frac{\partial}{\partial\theta}\log f(x\given\theta)\right)^2
	\given[\middle]\theta\right]
\end{align}
where $f(\cdot)$ is the conditioned likelihood function (density function)
and $x$ is the observed (realised) value of random variable $X$.

\section{Idea}

The idea is to measure ``how much information about a parameter $\theta$ can
we get from observing a random variable $X$.''
By information we mean ``how much can $\theta$ affect the distribution of $X$.''
% That is, a random variable $X$ might contain a certain amount of
% information about the parameter $\theta$, however, it is not clear to us how
% mush such parameter affects the distribution of $X$.
In the extreme cases, if the distribution of $X$ does not depend of $\theta$,
then no information about $\theta$ is gained from observing $X$;
if the distribution of $X$ is highly dependent of the parameter	$\theta$,
then a lot of information about $\theta$ is gained from observing $X$,
since different $\theta$'s would imply very differnet distributions of $X$.

\section{Explanation of the Formula}

To measure this, we use formula (\ref{eq:fisher}).
Essentially, we measure how much the density function will change if we nudge
the parameter $\theta$ by a little bit.
We use $\frac{\partial}{\partial\theta}\log f(x\given\theta)$ since we want 
to measure the ``percentage of change'' due to $\theta$.
However, we cannot simple calculate the expectation of $\frac{\partial}{\partial\theta}\log f(x\given\theta)$ since under some reularity conditions we have
\begin{align*}
	\E\left[\frac{\partial}{\partial\theta}\log f(x\given\theta)\given[\middle]\theta\right]
	&= \int_{\mathcal{X}} \frac{\partial}{\partial\theta}\Big(\log f(x\given\theta)\Big) f(x\given\theta) \dd{x} \\
	% &= \int_{\mathcal{X}} \frac1{f(x\given\theta)}\frac{\partial f(x\given\theta)}{\partial\theta} f(x\given\theta) \dd{x} \\
	&= \int_{\mathcal{X}} \frac{\partial}{\partial\theta}f(x\given\theta) \dd{x}
	= \frac{\partial}{\partial\theta} \int_{\mathcal{X}} f(x\given\theta) \dd{x}
	= 0.
\end{align*}
That is, the expected ``percentage of change'' is zero.
Thus, we measure the \emph{variance} of ``percentage of change'' to guage how much $\theta$ affects the distribution of $X$.
% Thus, the ``percentage of change'' is squared to ensure that the measure is not always zero.
Let $\ell_{x}(\theta):=\log f(x\given\theta)$ and
$\dot\ell_{x}(\theta):=\frac{\partial}{\partial\theta}\ell_{x}(\theta)$.
We have $\E{\dot\ell_{x}(\theta)}=0$ and $\var{\dot\ell_{x}(\theta)}=\mathcal{I}_{\theta}$.
Denote this result as $\dot\ell_{x}(\theta)\sim(0,\mathcal{I}_{\theta})$.

\section{Maximum Likelihood Estimation}

% Consider the log likelihood function $\log f(x\given\theta)$,
% denoted by $\ell_x(\theta)$.
% As shown above, we have that
% $\dot{\ell}_{x}(\theta):=\frac{\partial}{\partial\theta}\ell_{x}(\theta)$
% has expectation 0 and variance $\mathcal{I}_{\theta}$,
% i.e., $\dot{\ell}_{x}(\theta)\sim(0,\mathcal{I}_\theta)$.
If we estimate the parameter $\theta$ using MLE (using $\hat{\theta}_{\text{MLE}}$),
then it can be shown that
\begin{align}\label{eq:mle}
	\hat{\theta}_{\text{MLE}}
	\dto
	\normal{0,\mathcal{I}_{\theta}\inv},
\end{align}
that is, the bigger the Fisher information, the better the estimation.

\begin{proof}
	Let $x=\{x_1,...,x_n\}$ be iid samples from $f(x\given\theta_0)$ where
	$\theta_0$ is the true parameter.
	Let $\ell_{x}(\theta)$ denote $\sum_{i=1}^{n}\ell_{x_i}(\theta)$ be
	the likelihood function.

	We know that $\dot\ell_{x}(\theta)\sim(0,n\mathcal{I}_{\theta})$
	since $x_i$ are iid samples from $f(x\given\theta_0)$.
	We want to find the expectation of $\ddot\ell_{x}(\theta)$.
	Consider the second partial derivative of $\ell_{x}(\theta)$
	with respect to $\theta$:
	\begin{align*}
		\ddot\ell_{x}(\theta)
		:= \frac{\partial^2}{\partial\theta^2} \ell_{x}(\theta)
		= \frac{\ddot f(x\given\theta)}{f(x\given\theta)}
		- \left(\frac{\dot f(x\given\theta)}{f(x\given\theta)}\right)^2
	\end{align*}
	Take expectation of both sides and we have
	\begin{align*}
		\E{\ddot\ell_{x}(\theta)}
		= \E\left[ \frac{\ddot f(x\given\theta)}{f(x\given\theta)}
		- \left(\frac{\dot f(x\given\theta)}{f(x\given\theta)}\right)^2 \right]
		&= \E\left[\frac{\ddot f(x\given\theta)}{f(x\given\theta)}\right]
		- n\mathcal{I}_{\theta} \\
		&= \int_{\mathcal{X}}
		\frac{\ddot f(x\given\theta)}{f(x\given\theta)}f(x\given\theta) d{x}
		- n\mathcal{I}_{\theta} \\
		&= \underbrace{\int_{\mathcal{X}} \ddot f(x\given\theta) d{x}}_{=0}
		- n\mathcal{I}_{\theta} \\
		&= - n\mathcal{I}_{\theta}.
	\end{align*}
	where $\int_{\mathcal{X}}\ddot f(x\given\theta)d{x}=0$ is obtained by
	interchanging the integral and differentiation.
	Hence, $-\E{\ddot\ell_{x}(\theta)}=n\mathcal{I}_{\theta}$.

	Since $\hat{\theta}_{\text{MLE}}$ is an MLE estimator, 
	it satisfies $\dot\ell_{x}(\hat{\theta}_{\text{MLE}})=0$ by first order condition.
	By mean value theorem, we have
	\begin{align*}
		\underbrace{\dot\ell_{x}(\hat{\theta}_{\text{MLE}})}_{=0}
		- \dot\ell_{x}(\theta_0)
		= \ddot\ell_{x}(\bar\theta)(\hat{\theta}_{\text{MLE}}-\theta_0)
	\end{align*}
	where $\bar{\theta}$ is an value between $\theta_0$ and $\hat{\theta}_{\text{MLE}}$.
	By rearranging the terms we have
	\begin{align*}
		\sqrt{n}(\hat{\theta}_{\text{MLE}}-\theta_0)
		= -\left(\frac{\ddot\ell_{x}(\bar\theta)}{n}\right)^{-1}
		\frac{\dot\ell_{x}(\theta_0)}{\sqrt{n}}
		&\dto -(n\mathcal{I}_{\theta})^{-1}\normal{0,n\mathcal{I}_{\theta}} \\
		&= \normal{0,(n\mathcal{I}_{\theta})^{-1}}
	\end{align*}
	by Slutsky's Theorem
	where $({\ddot\ell_{x}(\bar\theta)}/{n})^{-1}\pto(n\mathcal{I}_{\theta})^{-1}$
	by weak law of large number and
	${\dot\ell_{x}(\theta_0)}/{\sqrt{n}}\dto\normal{0,n\mathcal{I}_{\theta}}$
	by central limit theorem.
\end{proof}

Notice that this results also applies to multi-dimension parameters $\vec{\theta}$
by replacing Fisher information by a Fisher information matrix and $\ddot\ell_{x}(\theta)$ by
the Hessian.

\section{Cram\'er-Rao Bounds}

One reason that MLE is favourable is that it is very efficient in the sense
that it has the smallest possible variance of all the estimators.
Cram\'er-Rao Bounds show that for any unbiased estimator $\tilde{\theta}$
of the parameter $\theta$, we have
\begin{align*}
	\var{\tilde{\theta}}\geq\frac1{n\mathcal{I}_{\theta}}.
\end{align*}
However, MLE are not always unbiased (e.g.\! MLE for variance).
But since the bias of MLE are on the order of $1/n$,
the bias is relatively small compared to its variance.

\begin{proof}
	Consider an unbiased estimator $\tilde{\theta}=t(x)$ of $\theta$.
	We have
	\begin{align*}
		\int_{\mathcal{X}} t(x)\dot\ell_{x}(\theta)f(x\given\theta) d{x}
		= \int_{\mathcal{X}} t(x)\dot f(x\given\theta) d{x}
		&= \frac{\partial}{\partial\theta} \int_{\mathcal{X}} t(x)f(x\given\theta) d{x} \\
		&= \frac{\partial}{\partial\theta} \theta
		= 1.
	\end{align*}
	Since $\dot\ell_{x}(\theta)$ has expectation zero,
	$\int_{\mathcal{X}}[t(x)-\theta]\dot\ell_{x}(\theta)f(x\given\theta)d{x}=1$.
	By applying Cauchy-Schwarz inequality, we have
	\begin{align*}
		\left[\int_{\mathcal{X}}[t(x)-\theta]\dot\ell_{x}(\theta)f(x\given\theta)d{x}\right]^2
		&\leq \int_{\mathcal{X}} [t(x)-\theta]^2 f(x\given\theta) d{x} \cdot
		\int_{\mathcal{X}} \dot\ell_{x}(\theta)^2 f(x\given\theta) d{x} \\
		\implies 1 &\leq \var{t(x)} \cdot \mathcal{I}_{\theta}
	\end{align*}
\end{proof}

\section{Observed Fisher Information}

In equation \eqref{eq:mle},
we derived the asymptotic distribution of MLE and found that
the variance of such distribution is the inverse of Fisher information.
However, in practice, it is hard to obtain Fisher information since it 
requires probability calculation.
Therefore, Fisher advocated of the use of \emph{Observed Fisher Information},
denoted by $I(x)$, defined by
\begin{align*}
	I(x)
	:= -\ddot\ell_{x}(\hat{\theta}_{\text{MLE}})
	= \left. \frac{\partial^2}{\partial\theta^2} \ell_{x}(\theta) \right|_{\hat{\theta}_{\text{MLE}}},
\end{align*}
which can be calculated numerically given the data $x$.
This is basically an application of the \emph{plug-in principle}.

Apart from being easy to calculate,
another benefit, argued by Fisher, of using the observed Fisher information 
is that it is a kind of \emph{conditioned inference}
(or in Fisher's jargon, approximate ancillary),
which bridges the gap between Bayesian and frequentist inference:
by using the ``observed'' information,
we are essentially updating our belief (or our confident) in the estimator based on the observed data,
which might be more relevant while making inferences.
\hfill$\blacksquare$

\vfill

\section*{References}

\begin{enumerate}[label = {[\arabic*]}]
	\item Wikipedia: \url{https://en.wikipedia.org/wiki/Fisher_information}
	\item Handbook of Econometrics. volume 4. p.2141. Whitney K. Newey \& Daniel Mcfadden.  
	\item Computer Age Statistical Inference. Chapter \emph{Fisherian Inference and Maximum Likelihood Estimation}. Bradley Efron \& Trevor Hastie.
\end{enumerate}

\end{document}
