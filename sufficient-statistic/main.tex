\documentclass[a4paper]{article}
% \usepackage[a4paper,vmargin={4cm,4cm},hmargin={4cm,4cm}]{geometry}
%<--------------------------------------------------------------------------->%
\usepackage{settings}
%<--------------------------------------------------------------------------->%
%%% Title %%%
\title{Sufficient Statistic}
\author{\href{https://jessekelighine.com}{jessekelighine.com}\\Jesse C.\ Chen\ 陳\,捷}
\date{\today}
%-----------------------------------------------------------------------------%

\begin{document}

\maketitle

\begin{definition}[Statistical Experiment]
	A \emph{statistical experiment} is a triple $(\mathcal{X},\mathcal{F},\{\Pr_{\theta}\}_{\theta\in\Theta})$ where
	$\mathcal{X}$ is the sample space,
	$\mathcal{F}$ is a $\sigma$-algebra on $\mathcal{X}$, and
	$\{\Pr_{\theta}\}_{\theta\in\Theta}$ is a collection of of probability measures
	on $\mathcal{X}$ parametrized by $\theta$ in the parameter space $\Theta$.
\end{definition}

\begin{definition}[Sufficient Statistic]
	Let $(\mathcal{X},\mathcal{F},\{\Pr_{\theta}\}_{\theta\in\Theta})$ be an statistical experiment.
	Let a measurable function $T:(\mathcal{X},\mathcal{F})\to(\mathcal{T},\mathcal{G})$ be a statistic.
	The statistic $T$ is said to be \emph{sufficient} if $\Pr_{\theta}(\cdot\given T)$ does not depend on $\theta$.
\end{definition}

\begin{example}\label{eg:bernoulli}
	Consider an experiment $(\mathcal{X},\mathcal{F},\{\Pr_{\theta}\}_{\theta\in[0,1]})$
	where $\mathcal{X}=\{0,1\}^{n}$,
	$\mathcal{F}$ is the power set of $\mathcal{X}$,
	and $\Pr_{\theta}$ is the joint distribution of $n$ iid $\bernoulli(\theta)$ distributions.
	Define statistic $T(x)\coloneqq\sum_{i=1}^{n}x_i$ where $x_i$ denotes the $i$-th component of $x$.
	We check that $T$ is a sufficient statistic:
	Given any $x\in\mathcal{X}$, let $t\coloneqq T(x)$ and we have
	\begin{align*}
		\pr[\theta]{X=x\given T=t}
		= \frac{ \pr[\theta]{X=x} }{ \pr[\theta]{T=t} }
		= \frac{ \prod_{i=1}^{n} \theta^{x_i} (1-\theta)^{1-x_i} } { {n\choose t}\theta^{t}(1-\theta)^{n-t} }
		= {n\choose t}\inv.
	\end{align*}
	Since the conditional distribution does not depend on $\theta$, $T$ is a sufficient statistic.
\end{example}

\begin{theorem}[Fisher-Neyman Factorization]\label{thm:fisher-neyman-factorization}
	Consider a statistical experiment $(\mathcal{X},\mathcal{F},\{\Pr_{\theta}\}_{\theta\in\Theta})$.
	A statistic $T:(\mathcal{X},\mathcal{F})\to(\mathcal{T},\mathcal{G})$ is a sufficient statistic iff
	there exists measurable functions $\{g_{\theta}\}_{\theta\in\Theta}$ defined on $(\mathcal{T},\mathcal{G})$ and
	$h$ defined on $(\mathcal{X},\mathcal{F})$ such that the probability density function can be decomposed as
	\begin{align}\label{eq:factorization}
		f(x\given\theta) = g_{\theta}(T(x)) h(x).
	\end{align}
\end{theorem}
\begin{proof}
	We only show the proof for discrete random variables.
	\footnote{
		For continuous random variables,
		one must deal with measure-theoretic technicalities in the proof.
		One can find a proof for general random variables in
		\emph{Chapter 2.2 Statistics and Sufficiency} from
		\cite{shao-1999}.
	}
	In this case, we have the probability density function $f(x\given\theta)=\pr[\theta]{X=x}$.
	\begin{itemize}
		\item[$(\Rightarrow)$]
			Suppose that $T$ is sufficient.
			Given any realization $x\in\mathcal{X}$,
			let $t\coloneqq T(x)$ and we have the following
			\begin{align*}
				\pr[\theta]{X=x}
				&= \pr[\theta]{X=x,T=t} \\
				&= \pr[\theta]{X=x\given T=t} \pr[\theta]{T=t} \\
				&= \underbrace{\pr{X=x\given T=t}}_{h(x)} \underbrace{\pr[\theta]{T=t}}_{g_{\theta}(t)}
			\end{align*}
			where the last equality is obtained by the definition of sufficiency.
		\item[$(\Leftarrow)$]
			Suppose we have the factorization \eqref{eq:factorization}.
			Given any realization $x\in\mathcal{X}$,
			let $t\coloneqq T(x)$ and we have
			\begin{align*}
				\pr[\theta]{X=x\given T=t}
				&= \frac{ \pr[\theta]{X=x,T=t} }{ \pr[\theta]{T=t} } \\
				&= \frac{ g_{\theta}(t) h(x) }{ \pr[\theta]{T=t} }
			\end{align*}
			where the second equality is obtained by using the factorization on the numerator.
			Now consider the denominator, we have
			\begin{align*}
				\pr[\theta]{T=t}
				= \sum_{x':T(x')=t} \pr[\theta]{X=x'}
				= \sum_{x':T(x')=t} g_{\theta}(t) h(x')
			\end{align*}
			where the second equality is obtained by factorization.
			Substitute the result back and we have
			\begin{align*}
				\pr[\theta]{X=x\given T=t}
				&= \frac{ g_{\theta}(t) h(x) }{\sum_{x':T(x')=t} g_{\theta}(t) h(x')} \\
				&= \frac{  h(x) }{\sum_{x':T(x')=t} h(x')}
			\end{align*}
			where the final expression does not depend on $\theta$.
			Hence $T$ is a sufficient statistic for $\theta$.
			\qedhere
	\end{itemize}
\end{proof}

\begin{example}[\autoref{eg:bernoulli} Continued]
	Now we check that $T(x)=\sum_{i=1}^{n}x_i$ is a sufficient statistic using
	\autoref{thm:fisher-neyman-factorization}:
	\begin{align*}
		\pr[\theta]{X=x}
		= \prod_{i=1}^{n} \theta^{x_i} (1-\theta)^{1-x_i}
		% = \theta^{\sum_{i=1}^{n}x_i} (1-\theta)^{n-\sum_{i=1}^{n}x_i}
		= \underbrace{\theta^{T(x)} (1-\theta)^{n-T(x)}}_{g_{\theta}(T(x))}.
	\end{align*}
	In this case, $h(x)=1$ is a constant function.
	Hence, $T$ is a sufficient statistic.
\end{example}

\begin{definition}[Sufficiency Principle]
	Let $(\mathcal{X},\mathcal{F},\{\Pr_{\theta}\}_{\theta\in\Theta})$ be a statistical experiment and
	let $T$ be a sufficient statistic.
	The \emph{sufficiency principle} states that inference about $\theta$ should depend only on $T$.
	That is, if two samples $x$ and $x'$ satisfies $T(x)=T(x')$,
	then they should lead to the same inference about $\theta$.
\end{definition}

\begin{remark}
	The concept of sufficient statistic and the sufficiency principle
	are both due to Sir Ronald Fisher in the early 20th century.
\end{remark}

\printbibliography

\end{document}
